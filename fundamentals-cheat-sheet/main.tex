% !TEX program = xelatex
\documentclass[10pt,b5paper,openany]{book} %openany: 新章节不强制从奇数页开始

% 调整页边距
\usepackage{geometry}
\geometry{right=1.6cm,left=1.6cm,top=2.3cm,bottom=2cm}

% 调整行距
\usepackage{setspace}

% 设置英文字体
\usepackage[T1]{fontenc}
\usepackage{charter}

% 漂亮的数学字体
\usepackage{mathpazo} % psnfss 字体宏集之一, Palatino 风格

% ams大礼包
\usepackage{amsmath}
\usepackage{amssymb}
\usepackage{amsfonts}

% slashed letters
\usepackage{cancel}

% 删除线 波浪线 斜删除线 双下划线
\usepackage{ulem}

% 输入数组矩阵
\usepackage{array}

% 输入正体希腊字母
\usepackage{upgreek}

% 调整参考文献引用格式
\usepackage{cite}

% 超链接
\usepackage[colorlinks, linkcolor=black]{hyperref}

% 插入图片等
\usepackage{graphicx}
\graphicspath{
  {./image/}
}

% 浮动体标题的格式
\usepackage{caption}

% 给符号加粗用
\usepackage{bm}

% 命令
% 长横线
\newcommand{\HRule}{\rule{\linewidth}{0.5mm}} 

% 评论使用的框
\newcommand{\mybox}[1]{ 
  \begin{center}
    \framebox[1.15\width][s]{
      \parbox[t]{32em}{

        \mbox{}

        #1

        \mbox{} \\[-0.8cm]

      }
    }
  \end{center}
}

% 无标号chapter
\newcommand{\nonumchap}[1]{
\phantomsection % 加这个命令后, 目录中的超链接才指向正确的页码
\chapter*{#1} %开始一段不带编号的章
\addcontentsline{toc}{chapter}{#1} %使目录中以章级别显示
}

% 无标号section
\newcommand{\nonumsec}[1]{
\phantomsection 
\section*{#1} 
\addcontentsline{toc}{section}{#1}
}

% 无标号subsection
\newcommand{\nonumssec}[1]{
\phantomsection 
\subsection*{#1} 
\addcontentsline{toc}{subsection}{#1}
}

\begin{document}

\begin{titlepage}
  \centering

  \mbox{}

  {\Huge \textbf{Quick Cheat Sheet}}

  (for fundamentals of physics) \\[0.8cm]

  \HRule \\[0.9cm]

  {\Large  ReEFT} \\[0.3cm]

  {\small \today} \\ [4.6cm]

  {\Huge $\bm{Mechanics}$} \\ [1.6cm]

  {\Huge $\bm{Electrodynamics}$} \\ [1.6cm]

  {\Huge $\bm{Quantum\ Mechanics}$} \\ [1.6cm]

  {\Huge $\bm{Statistical\ Mechanics}$} \\ [1.6cm]

\end{titlepage}

\pagenumbering{roman}
\setcounter{page}{1}

\tableofcontents
\addcontentsline{toc}{chapter}{Contents}
\clearpage

\begin{spacing}{1.2}
\pagenumbering{arabic}
\setcounter{page}{1}

\part{Mechanics}

\chapter{Hamilton's Principle}

Define the \textit{action} of a system of particles as: 
\begin{equation}
  S = \int_{t_1}^{t_2} L(q, \dot{q}, t)dt
\end{equation}

\chapter{Lagrange's Formulation}

\chapter{Hamilton's Formulation}


\part{Electrodynamics}

\part{Quantum Mechanics}

\chapter{Formalism}

\part{Statistical Mechanics}


\end{spacing}
\end{document}