% !TEX encoding = UTF-8
% !TEX program = xelatex
\documentclass[10pt,b5paper, openany]{book} %openany: 新章节不强制从奇数页开始

% 调整页边距
\usepackage{geometry}
\geometry{right=1.6cm,left=1.6cm,top=2.3cm,bottom=2cm}

% 调整行距
\usepackage{setspace}

% 设置中文字体
\usepackage[UTF8, fontset=none]{ctex}
\usepackage{fontspec}
\xeCJKDeclareCharClass{CJK}{%  %使Unicode带圈字符正确显示
  "24EA,           %
  "2460->"2473,    %
  "3251->"32BF,    %
  "24FF,           %
  "2776->"277F,    %
  "24EB->"24F4     %
}
\setCJKmainfont[BoldFont = 思源黑体, ItalicFont = 华文楷体]{冬青黑体简体中文}
\setCJKmonofont{兰亭黑-简}

% 漂亮的数学字体
\usepackage{mathpazo} % psnfss 字体宏集之一, Palatino 风格
% 设置英文字体
\usepackage[T1]{fontenc}
\usepackage{charter}

% ams大礼包
\usepackage{amsmath}
\usepackage{amssymb}
\usepackage{amsfonts}

% slashed letters
\usepackage{cancel}

% 删除线 波浪线 斜删除线 双下划线
\usepackage{ulem}
% \sout{文字} %删除线
% \uwave{文字} %波浪线
% \xout{文字} %斜删除线
% \uuline{文字}  %双下划线

% 输入数组矩阵
\usepackage{array}

% 输入正体希腊字母
\usepackage{upgreek}

% 调整参考文献引用格式
\usepackage{cite}

% 超链接
\usepackage[colorlinks, linkcolor=black, urlcolor=cyan]{hyperref}

% 插入图片等
\usepackage{graphicx}
\graphicspath{
  {./image/}
}

% 浮动体标题的格式
\usepackage{caption}

% 给符号加粗用
\usepackage{bm}

% 调整 pdf 大纲格式
\usepackage{hyperref}
\hypersetup{
    bookmarksnumbered=true, 
    % bookmarksdepth=section,
}

% 调整页眉章节标题样式
\usepackage{fancyhdr}
\fancypagestyle{general}{
  \fancyhf{} % 清空当前页眉和页脚的设置

  % 设置页眉
  \fancyhead[LE]{{\bfseries \thepage}\quad\quad\nouppercase{\leftmark}}
  \fancyhead[RE]{}
  \fancyhead[RO]{\nouppercase{\rightmark}\quad\quad{\bfseries \thepage}}
  \fancyhead[LO]{}
}

% 命令
% 长横线
\newcommand{\HRule}{\rule{\linewidth}{0.5mm}} 

% 评论使用的框
\newcommand{\mybox}[1]{ 
  \begin{center}
    \setlength{\fboxrule}{1pt}
    \framebox[1.15\width][s]{
      \parbox[t]{32em}{

        \mbox{}

        #1

        \mbox{} \\[-0.8cm]

      }
    }
  \end{center}
}

% 着重号
\usepackage{xeCJKfntef}
\xeCJKsetup{underdot/symbol={\normalfont^^b7}}
\newcommand{\cemph}[1]{\CJKunderdot{#1}}
% \newcommand{\cemph}[1]{#1}

% 无标号 chapter (改变页眉页脚格式)
\newcommand{\nonumchap}[1]{
  \fancypagestyle{nonumchap}{
    \fancyhead[LE]{{\bfseries \thepage}\quad\quad{#1}}
  }
  \pagestyle{nonumchap}
  \phantomsection % 加这个命令后, 目录中的超链接才指向正确的页码
  \chapter*{#1} %开始一段不带编号的章
  \addcontentsline{toc}{chapter}{#1} %使目录中以章级别显示
}

% 无标号 chapter (保持页眉页脚格式)
\newcommand{\nonumchapkphdr}[1]{
  \phantomsection % 加这个命令后, 目录中的超链接才指向正确的页码
  \chapter*{#1} %开始一段不带编号的章
  \addcontentsline{toc}{chapter}{#1} %使目录中以章级别显示
}

% 无标号 section (改变页眉页脚格式)
\newcommand{\nonumsec}[1]{
  \fancypagestyle{nonumsec}{
    \fancyhead[RO]{{#1}\quad\quad{\bfseries \thepage}}
  }
  \pagestyle{nonumsec}
  \phantomsection 
  \section*{#1} 
  \addcontentsline{toc}{section}{#1}
}

% 无标号 section (保持页眉页脚格式)
\newcommand{\nonumseckphdr}[1]{
  \phantomsection 
  \section*{#1} 
  \addcontentsline{toc}{section}{#1}
}

% 无标号 subsection
\newcommand{\nonumssec}[1]{
  \phantomsection 
  \subsection*{#1} 
  \addcontentsline{toc}{subsection}{#1}
}

\begin{document}
\pagestyle{fancy}

\begin{titlepage}
  \centering

  \mbox{} \\[-0.3cm]

  {\bfseries \fontsize{32pt}{1pt}\selectfont \xeCJKsetup{CJKglue=\hskip 1.5pt}强子物理笔记} \\[0.4cm]

  {\bfseries \fontsize{16pt}{1pt}\selectfont from LQCD to Hadrons} \\[0.6cm]

  \HRule \\[0.8cm]

  {\Large Tinikov} \\[0.3cm]

  {\small 创建于2022年11月1日} \\[-0.1cm]

  {\small 最新修改于\today} \\[0.4cm]
\end{titlepage}

\pagenumbering{roman}
\setcounter{page}{1}
\nonumchap{前言}

\begin{spacing}{1.8}
  参考书目: 
  \begin{itemize}
    \item \textit{Quantum Chromodynamics on the Lattice: An Introductory Presentation} - Gattringer, Lang
    \item \textit{An Introductory to Quantum Field Theory} - Peskin, Schroeder
    \item \textit{Lie Algebras in Particle Physics} - Georgi
  \end{itemize}

  如发现任何错误或遗漏, 或希望与我讨论书中的内容, 欢迎通过\href{mailto:tinikov137@gmail.com}{邮件}联系.

  我的邮箱: \texttt{tinikov137@gmail.com}
\end{spacing}
\pagestyle{general}
\clearpage

\phantomsection
\addcontentsline{toc}{chapter}{目录}
\tableofcontents
\clearpage

\begin{spacing}{1.2}
  \pagenumbering{arabic}
  \setcounter{page}{1}

  \pagestyle{general}
  \chapter{Quark Model}

\section{}

  \pagestyle{general}
  \chapter{Lattice QCD}

The QCD Lagrangian in the continuum:
\begin{equation}
    \mathcal{L}=\overline{\psi}(i\cancel{D}-m)\psi - \                                                                                                                                                                                                                                 frac{1}{4}(F^a_{\mu\nu})^2,
\end{equation}
where
\begin{gather}
    F^a_{\mu\nu}=\partial_\mu A_\nu^a - \partial_\nu A_\mu^a + gf^{abc}A_\mu^b A_\nu^c, \\
    D_\mu = \partial_\mu - ig A_\mu^aT^a.
\end{gather}
An explicit representation of $T^a$: 
\begin{equation}
    T^a = \frac{\lambda_a}{2},
\end{equation}
wherein the $\lambda_a(a=1\dots 8)$ are the Gell-Mann matrices.

\section{Constructing the Euclidean Correlators}

  % \pagestyle{general}
  % \chapter{Bethe-Salpeter Wave Function}

\section{}

\end{spacing}
\end{document}